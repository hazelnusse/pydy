\documentclass[letterpaper,11pt]{article}
\usepackage[margin=1in,centering]{geometry}
\usepackage{fancyhdr}
\usepackage{amsmath}
\usepackage{amssymb}
\newcommand{\bs}[1]{ \boldsymbol{ #1 } }
\newcommand{\ud}{\mathrm{d}}
\pagestyle{fancy}
\fancyhead[L]{Benchmark bicycle model equations of motion, Dale L. Peterson}
\fancyhead[R]{\today}
\fancyfoot[L,R]{}
\fancyfoot[C]{\thepage}
\begin{document}
\begin{abstract}

  Nonlinear equations of motion for a benchmark bicycle are presented in
  symbolic form.  They are validated numerically against Meijaard et al. and
  Basu-Mandal et al.  Numerical results match to at least 14 decimal places.

\end{abstract}

\section*{Model description}
Consider the following four rigid bodies: a rear wheel $C$, a frame with
rigidly attached rider $D$, a front fork/handlebar assembly $E$, and a front
wheel $F$.
The following eleven (non-minimal) configuration coordinates are defined:
\begin{table}[h]
  \begin{center}
    \begin{tabular}{rl}
      $q_1:$ & Rear frame yaw \\
      $q_2:$ & Rear frame lean \\
      $q_3:$ & Rear frame pitch \\
      $q_4:$ & Rear wheel angle relative to rear frame \\
      $q_5:$ & Steer angle relative to rear frame \\
      $q_6:$ & Front wheel angle relative to front fork \\
      $q_7:$ & Rear wheel contact point position in $\bs{n}_1$ direction \\
      $q_8:$ & Rear wheel contact point position in $\bs{n}_2$ direction \\
      $q_9:$ & Front frame yaw relative to rear frame \\
      $q_{10}:$ & Front frame lean \\
      $q_{11}:$ & Front frame pitch
    \end{tabular}
  \end{center}
\end{table}

The bicycle frame with rigidly attached rider $D$ are oriented with Euler 3-1-2
($q_1, q_2, q_3$) angles (identical to the ($\psi, \phi, \theta_B$) Euler
angles in \cite{Meijaard2007}).  The rear wheel $R$ is oriented relative to $D$
by an angle $q_4$ ($\theta_R$ in \cite{Meijaard2007}), the front fork relative
to $D$ by an angle $q_5$ ($\phi$ in \cite{Meijaard2007}), and the front wheel
relative to the front fork $E$ by an angle $q_6$ ($\theta_F$ in
\cite{Meijaard2007}).  The last three configuration coordinates are dependent
upon $q_2$, $q_3$, and $q_5$ in relatively simple ways.  In contrast, the
relationship between rear frame lean $q_2$, rear frame pitch $q_3$ and the
steer $q_5$, is relatively complex \cite{Peterson2008a}.  For these reasons,
rear frame lean and steer will be treated as independent, rear frame pitch
$q_4$ will be determined numerically by solving the associated holonomic
constraint equation, and {\it{then}} analytic expressions for determining
$q_9$, $q_{10}$, and $q_{11}$ can be used.

Explicit direction cosine matrices are provided in the
appendix.

\section*{Kinematic constraints}
The use of a non-minimal set of coordinates necessitates the discussion of
holonomic constraints associated with the extra generalized coordinates.  The
relationship between the rear frame lean, rear frame pitch, and the steer is:
\begin{align*}
0 & =  \bs{r}^{FN/CN} \cdot \bs{n}_3 \\
  & = r_{ft} + l_s s_2 + l_f (s_2 s_5 - c_2 s_3 c_5) +
    r_f \sqrt{1 - (s_2 c_5 + c_2 s_3 s_5)^2} - r_{rt} - r_r c_2 - l_r c_2 s_3
  \label{eq:holonomic}
\end{align*}
Put simply, this equation requires that the vector from the rear wheel ground
contact to the front wheel ground contact have no component out of the ground
plane.  Given $q_2$ and $q_5$, this equation is readily solved numerically for
the rear frame pitch $q_3$ \cite{Peterson2008a}.

The relationships between the three other dependent configuration coordinates
$q_9$, $q_{10}$, and $q_{11}$ can now be expressed conveniently in terms of
$q_2$, $q_3$, and $q_5$.  The

The unit vector lying in the plane of the front wheel that
can be thought of as pointing from the front wheel center torwards the front
contact point is formed by taking the projection of the $\bs{n}_3$ unit vector
onto the plane of the front wheel, and normalizing to make it unit length:
\begin{align*}
  \bs{g}_3 & = (\bs{n}_3 - (\bs{n}_3 \cdot \bs{e}_2) \bs{e}_2) / |(\bs{n}_3 -
  (\bs{n}_3 \cdot \bs{e}_2) \bs{e}_2)| \\
  & = ((s_2 s_5 - c_2 s_3 c_5) \bs{e}_1 + c_2 c_3 \bs{e}_3) / \sqrt{(s_2 s_5 -
  c_2 s_3 c_5)^2 + (c_2 c_3)^2}
\end{align*}
The vector aligned with the front wheel track is:
\begin{align*}
  \bs{g}_1 & = \bs{e}_2 \times \bs{g}_3 \\
  & = \left(c_2 c_3 \bs{e}_1 + (c_2 s_3 c_5 - s_2 s_5) \bs{e}_3\right) / \sqrt{(s_2 s_5 -
  c_2 s_3 c_5)^2 + (c_2 c_3)^2} \\
  & = \left((c_2 c_5 - s_2 s_3 s_5) \bs{a}_1 + c_3 s_5 \bs{a}_2\right) / \sqrt{(s_2 s_5 -
  c_2 s_3 c_5)^2 + (c_2 c_3)^2}
\end{align*}

Given rear frame lean $q_2$, rear frame pitch $q_3$, and the steer angle
$q_5$, the above vectors can be used to furnish the remaining holonomic
constraints:
\begin{align*}
  q_9 & = \arcsin(\bs{g}_1 \cdot \bs{a}_2) = \arcsin(c_3 s_5) \\
  q_{10} & = \arcsin(\bs{e}_2 \cdot \bs{n}_3) = \arcsin(s_2 c_5 + c_2 s_3 s_5)
  q_{11} & = \arcsin(\bs{e}_3 \cdot \bs{g}_1) = \arcsin(c_2 s_3 c_5 - s_2 s_5)
\end{align*}

For purpose of determining ground reaction forces, the unit vectors parallel to
the the ground plane that are aligned with, and perpendicular to, the front and
rear wheel 

\begin{tabular}{rcc}
  & Rear wheel & Front wheel \\
  Tangent to path: & $\bs{a}_1$ & $\left( (c_2 c_5 - s_2 s_3 s_5)\bs{a}_1
  + c_3 s_5 \bs{a}_2 \right) /\sqrt{(s_2 s_5 - c_2 s_3 c_5)^2 + (c_2 c_3)^2}$ \\
  Normal to path: & $\bs{a}_2$ & $\left( - c_3 s_5 \bs{a}_1
  + (c_2 c_5 - s_2 s_3 s_5) \bs{a}_2 \right) /\sqrt{(s_2 s_5 - c_2 s_3 c_5)^2 + (c_2 c_3)^2}$
\end{tabular}


\section*{Motion constraints}
Define six generalized speeds:
\begin{align*}
  u_i & \triangleq \bs{\omega}^D \cdot \bs{d}_i \qquad (i = 1, 2, 3)\\
  u_4 & \triangleq \bs{\omega}^C \cdot \bs{d}_2\\
  u_5 & \triangleq \bs{\omega}^E \cdot \bs{d}_3\\
  u_6 & \triangleq \bs{\omega}^F \cdot \bs{e}_2
\end{align*}

These generalized speed definitions follow the guidelines of \cite{Mitiguy1996}
for generating `efficient' equations of motion.  As it turns out, they also
simplify the formulation of the motion constraints imposed by pure rolling.  As
will be made clear in the next section, only three of these six generalized
speeds can be treated as independent.  The resulting angular velocities (with
respect to the inertial frame) of each of the four rigid bodies are:
\begin{align*}
  \bs{\omega}^D & = u_1 \bs{d}_1 + u_2 \bs{d}_2 + u_3 \bs{d}_3 \\
  \bs{\omega}^C & = u_1 \bs{d}_1 + u_4 \bs{d}_2 + u_3 \bs{d}_3 \\
  \bs{\omega}^E & = u_1 \bs{d}_1 + u_2 \bs{d}_2 + u_5 \bs{d}_3 \\
                & = (c_5 u_1 + s_5 u_2)\bs{e}_1 + (-s_5 u_1 + c_5 u_2) \bs{e}_2
                    + u_5 \bs{e}_3 \\
  \bs{\omega}^F & = (c_5 u_1 + s_5 u_2)\bs{e}_1 + u_6 \bs{e}_2
                    + u_5 \bs{e}_3
\end{align*}


The nonholonomic constraints may be written by formulating the velocity of the
top of the steer axis $DE$ in two equivalent ways.  The first approach makes
use of the rear wheel and rear frame, while the second approach makes use of
the front wheel and frame; the constraints arise by equating these two
expressions for the velocity of $DE$.
\begin{align*}
  \bs{v}^{DE}_r & = \bs{\omega}^C \times \bs{r}^{C^* / CN} + \bs{\omega}^D
      \times \bs{r}^{DE / C^*} \\
      & = -r_r c_3 u_4 \bs{d}_1 + (l_r u_3 + r_r c_3 u_1 + r_r s_3 u_3)\bs{d}_2
      - (l_r u_2 + r_r s_3 u_4)\bs{d}_3 \\
    \bs{v}^{DE}_f & = \bs{\omega}^F \times \bs{r}^{F^* / FN} + \bs{\omega}^E
      \times \bs{r}^{DE / F^*} \\
      & = (-l_s (c_5 u_2 - s_5 u_1) - g_{31} u_6) \bs{e}_1 +
      (l_s (c_5 u_1 + s_5 u_2) + (c_5 u_1 + s_5 u_2) g_{31} - l_f u_5 -
        g_{31} u_5) \bs{e}_2 \\
        & \quad + (l_f (c_5 u_2 - s_5 u_1) + g_{31} u_6) \bs{e}_3
\end{align*}
Equating these two expressions and resolving into components ($\bs{d}_i \cdot
(\bs{v}^{DE}_r - \bs{v}^{DE}_f) = 0$) allows for the constraints to be written
in matrix form as:
\begin{align*}
  \left[
  \begin{array}{cccccc}
    c_5 g_{31} s_5 & l_s + s_5^2 g_{31} & 0 & -r_r c_3 & -l_f s_5 - g_{31} s_5
    & c_5 g_{31} \\
    -l_s + r_r c_3 - c_5^2 g_{31} & -c_5 g_{31} s_5 & l_r + r_r s_3 & 0 & l_f
    c_5 + c_5 g_{31} & g_{31} s_5 \\
    l_f s_5 & -l_r - l_f c_5 & 0 & -r_r s_3 & 0 & -g_{31}
  \end{array}
  \right]
  \left[
    \begin{array}{c}
      u_1 \\ u_2 \\ u_3 \\ u_4 \\ u_5 \\ u_6
    \end{array}
    \right]
  & =
  \left[
    \begin{array}{c}
      0 \\ 0 \\ 0
    \end{array}
    \right]
\end{align*}

By choosing three of the six generalized speeds as independent, the constraint
equations can be rewritten and solved for the dependent speeds:
\begin{align*}
  B u & = 0 \\
  B_d u_d + B_i u_i & = 0 \\
  u_d & = -B_d^{-1} B_i u_i \\
      & = T u_i
\end{align*}
Where $B_d$ and $B_i$ are formed by choosing the columns of $B$ corresponding
to the index (subscript) of the dependent and independent generalized speeds.

Given any three of the generalized speeds, the above constraints provide the
means to determine the three.  The resulting kinematic differential equations
(involving {\it{all}} generalized speeds) are shown in Table \ref{kindiffs}
Note that expressions for $\dot{q}_7$ and $\dot{q}_8$ in Table \ref{kindiffs}
are implicitly defined in terms of $\dot{q}_3$ and $\dot{q}_4$ for brevity.

\begin{table}[!h]
  \begin{center}
    \begin{tabular}{rclcrcl}
        $\dot{q}_1$ & $=$ & $-(s_3/c_2) u_1 + (c_3/c_2) u_3$ &$\quad$&
        $\dot{q}_5$ & $=$ & $u_5 - u_3$ \\
        $\dot{q}_2$ & $=$ & $c_3 u_1 + s_3 u_3$ &$\quad$&
        $\dot{q}_6$ & $=$ & $s_5 u_1 - c_5 u_2 + u_6$ \\
        $\dot{q}_3$ & $=$ & $s_3 t_2 u_1 + u_2 - c_3 t_2 u_3$ &$\quad$&
        $\dot{q}_7$ & $=$ & $-r_r c_1 (\dot{q}_3 + \dot{q}_4)$ \\
        $\dot{q}_4$ & $=$ & $u_4 - u_2$ & $\quad$ &
        $\dot{q}_8$ & $=$ & $-r_r s_1 (\dot{q}_3 + \dot{q}_4)$
    \end{tabular}
  \end{center}
  \caption{Kinematic differential equations}
  \label{kindiffs}
\end{table}

\bibliographystyle{plain}
\bibliography{../../../../../Documents/references}

\end{document}
