\documentclass[letterpaper,11pt]{article}
\usepackage[margin=1in,centering]{geometry}
\usepackage{verbatim}
\usepackage{fancyhdr}
\usepackage{amsmath}
\usepackage{amssymb}
\newcommand{\bs}[1]{ \boldsymbol{ #1 } }
\newcommand{\ud}{\mathrm{d}}
\pagestyle{fancy}
\fancyhead[L]{Benchmark bicycle model equations of motion, Dale L. Peterson}
\fancyhead[R]{\today}
\fancyfoot[L,R]{}
\fancyfoot[C]{\thepage}
\begin{document}
\begin{abstract}

  Nonlinear equations of motion for a benchmark bicycle are presented in
  symbolic form.  They are validated numerically against Meijaard et al. and
  Basu-Mandal et al.  Numerical results match to at least 14 decimal places.

\end{abstract}

\section*{Model description}
Consider the following four rigid bodies: a rear wheel $C$, a frame with
rigidly attached rider $D$, a front fork/handlebar assembly $E$, and a front
wheel $F$.
The following eleven (non-minimal) configuration coordinates are defined:
\begin{table}[h]
  \begin{center}
    \begin{tabular}{rl}
      $q_1:$ & Rear frame yaw \\
      $q_2:$ & Rear frame lean \\
      $q_3:$ & Rear frame pitch \\
      $q_4:$ & Rear wheel angle relative to rear frame \\
      $q_5:$ & Steer angle relative to rear frame \\
      $q_6:$ & Front wheel angle relative to front fork \\
      $q_7:$ & Rear wheel contact point position in $\bs{n}_1$ direction \\
      $q_8:$ & Rear wheel contact point position in $\bs{n}_2$ direction \\
      $q_9:$ & Front frame yaw relative to rear frame \\
      $q_{10}:$ & Front frame lean \\
      $q_{11}:$ & Front frame pitch
    \end{tabular}
  \end{center}
\end{table}

The bicycle frame with rigidly attached rider $D$ are oriented with Euler 3-1-2
($q_1, q_2, q_3$) angles (identical to the ($\psi, \phi, \theta_B$) Euler
angles in \cite{Meijaard2007}).  The rear wheel $R$ is oriented relative to $D$
by an angle $q_4$ ($\theta_R$ in \cite{Meijaard2007}), the front fork relative
to $D$ by an angle $q_5$ ($\phi$ in \cite{Meijaard2007}), and the front wheel
relative to the front fork $E$ by an angle $q_6$ ($\theta_F$ in
\cite{Meijaard2007}).  The last three configuration coordinates are dependent
upon $q_2$, $q_3$, and $q_5$ in relatively simple ways.  In contrast, the
relationship between rear frame lean $q_2$, rear frame pitch $q_3$ and the
steer $q_5$, is relatively complex \cite{Peterson2008a}.  For these reasons,
rear frame lean and steer will be treated as independent, rear frame pitch
$q_4$ will be determined numerically by solving the associated holonomic
constraint equation, and {\it{then}} analytic expressions for determining
$q_9$, $q_{10}$, and $q_{11}$ can be used.

Explicit direction cosine matrices are provided in the
appendix.

\section*{Kinematic constraints}
The use of a non-minimal set of coordinates necessitates the discussion of
holonomic constraints associated with the extra generalized coordinates.  The
relationship between the rear frame lean, rear frame pitch, and the steer is:
\begin{align*}
0 & =  \bs{r}^{FN/CN} \cdot \bs{n}_3 \\
  & = r_{ft} + l_s s_2 + l_f (s_2 s_5 - c_2 s_3 c_5) +
    r_f \sqrt{1 - (s_2 c_5 + c_2 s_3 s_5)^2} - r_{rt} - r_r c_2 - l_r c_2 s_3
  \label{eq:holonomic}
\end{align*}
Put simply, this equation requires that the vector from the rear wheel ground
contact to the front wheel ground contact have no component out of the ground
plane.  Given $q_2$ and $q_5$, this equation is readily solved numerically for
the rear frame pitch $q_3$ \cite{Peterson2008a}.

The relationships between the three other dependent configuration coordinates
$q_9$, $q_{10}$, and $q_{11}$ can now be expressed conveniently in terms of
$q_2$, $q_3$, and $q_5$.  The

The unit vector in the plane of the front wheel that can be thought of as
pointing from the front wheel center to the center of the tire casing directly
above the contact point is formed by taking the projection of the $\bs{n}_3$
unit vector onto the plane of the front wheel, and normalizing to make it unit
length:
\begin{align*}
  \bs{g}_3 & = (\bs{n}_3 - (\bs{n}_3 \cdot \bs{e}_2) \bs{e}_2) / |(\bs{n}_3 -
  (\bs{n}_3 \cdot \bs{e}_2) \bs{e}_2)| \\
  & = ((s_2 s_5 - c_2 s_3 c_5) \bs{e}_1 + c_2 c_3 \bs{e}_3) / \sqrt{(s_2 s_5 -
  c_2 s_3 c_5)^2 + (c_2 c_3)^2}
\end{align*}
The unit vector parallel to the intersection of the front wheel plane and the
ground plane:
\begin{align*}
  \bs{g}_1 & = \bs{e}_2 \times \bs{g}_3 \\
  & = \left(c_2 c_3 \bs{e}_1 + (c_2 s_3 c_5 - s_2 s_5) \bs{e}_3\right) / \sqrt{(s_2 s_5 -
  c_2 s_3 c_5)^2 + (c_2 c_3)^2} \\
  & = \left((c_2 c_5 - s_2 s_3 s_5) \bs{a}_1 + c_3 s_5 \bs{a}_2\right) / \sqrt{(s_2 s_5 -
  c_2 s_3 c_5)^2 + (c_2 c_3)^2}
\end{align*}
The unit vector perpendicular to the intersection of the front wheel plane and
the ground plane is:
\begin{align*}
  \bs{h}_2 & = \bs{n}_3 \times \bs{g}_1 \\
  & = \left(-c_3 s_5 \bs{a}_1 + (c_2 s_5 - s_2 s_3 s_5) \bs{a}_2\right) / \sqrt{(s_2 s_5 -
  c_2 s_3 c_5)^2 + (c_2 c_3)^2}
\end{align*}
When the bicycle is travelling forward, $\bs{g}_1$ points forward and $\bs{h}_2$ points to
the rider's right.

Given rear frame lean $q_2$, rear frame pitch $q_3$, and the steer angle
$q_5$, the above vectors can be used to furnish the remaining holonomic
constraints:
\begin{align*}
  q_9 & = \arcsin(\bs{g}_1 \cdot \bs{a}_2) = \arcsin\left(c_3 s_5 / \sqrt{(s_2 s_5 -
  c_2 s_3 c_5)^2 + (c_2 c_3)^2}\right) \\
  q_{10} & = \arcsin(\bs{e}_2 \cdot \bs{n}_3) = \arcsin\left(s_2 c_5 + c_2 s_3
  s_5\right)
  \\
  q_{11} & = \arcsin(\bs{e}_3 \cdot \bs{g}_1) = \arcsin \left( (c_2 s_3 c_5 - s_2
    s_5) /\sqrt{(s_2 s_5 - c_2 s_3 c_5)^2 + (c_2 c_3)^2}\right)
\end{align*}

For purpose of determining ground reaction forces, the unit vectors parallel to
the the ground plane that are aligned with, and perpendicular to, the front and
rear wheel lean frames are:
\begin{tabular}{rcc}
  & Rear wheel & Front wheel \\
  Tangent to path: & $\bs{a}_1$ & $\bs{g}_1$\\
  Normal to path: & $\bs{a}_2$ & $\bs{h}_2$
\end{tabular}

\section*{Motion constraints}
Define six generalized speeds:
\begin{align*}
  u_i & \triangleq \bs{\omega}^D \cdot \bs{d}_i \qquad (i = 1, 2, 3)\\
  u_4 & \triangleq \bs{\omega}^C \cdot \bs{d}_2\\
  u_5 & \triangleq \bs{\omega}^E \cdot \bs{d}_3\\
  u_6 & \triangleq \bs{\omega}^F \cdot \bs{e}_2
\end{align*}

These generalized speed definitions follow the guidelines of \cite{Mitiguy1996}
for generating efficient equations of motion.  As it turns out, they also
simplify the formulation of the motion constraints imposed by pure rolling.  As
will be made clear in the next section, only three of these six generalized
speeds can be treated as independent.  The resulting angular velocities (with
respect to the inertial frame) of each of the four rigid bodies are:
\begin{align*}
  \bs{\omega}^D & = u_1 \bs{d}_1 + u_2 \bs{d}_2 + u_3 \bs{d}_3 \\
  \bs{\omega}^C & = u_1 \bs{d}_1 + u_4 \bs{d}_2 + u_3 \bs{d}_3 \\
  \bs{\omega}^E & = u_1 \bs{d}_1 + u_2 \bs{d}_2 + u_5 \bs{d}_3 \\
                & = (c_5 u_1 + s_5 u_2)\bs{e}_1 + (-s_5 u_1 + c_5 u_2) \bs{e}_2
                    + u_5 \bs{e}_3 \\
  \bs{\omega}^F & = (c_5 u_1 + s_5 u_2)\bs{e}_1 + u_6 \bs{e}_2
                    + u_5 \bs{e}_3
\end{align*}
The rear wheel and the rear frame together comprise a gyrostat, as do the front frame
and front wheel; their inertia components are time independent constants
\cite{Wittenburg1977}.  As a result, we only concern ourselves with the two
points, the rear assembly mass center and the front assembly mass center.  The
velocities of these two points are:
\begin{align*}
  \bs{v}^{CDO} & = \bs{\omega}^C \times \bs{r}^{CO/CN} + \bs{\omega}^D \times
  \bs{r}^{CDO/CO}\\
  & = (l_2 u_2 + r_{rt} s_2 u_3 - c_3 (r_r + c_2 r_{rt}) u_4) \bs{d}_1 \\
  & \quad +
  ( (c_3 (r_r + c_2 r_{rt}) - l_2) u_1 + (l_1 + s_3 (r_r + c_2 r_{rt})) u_3) \bs{d}_2 \\
  & \quad -
  (s_2 r_{rt} u_1 + l_1 u_2 + s_3 (r_r + c_2 r_{rt}) u_4) \bs{d}_3\\
  \bs{v}^{EFO} & = \bs{\omega}^F \times \bs{r}^{FO/FN} + \bs{\omega}^E \times
  \bs{r}^{EFO/FO}\\
  & =\\
\end{align*}

The choice to treat the rear and front assemblies as gyrostats is justified by
the simplifications and computational gains that are described in
\cite{Mitiguy2001}.  The conversion of the parameters described in
\cite{Meijaard2007} to the parameters (both inertial and geometric) use here
are described in the appendix.

The nonholonomic constraints are formed by requiring that the velocity of point
$DE$ be equal when it is formed through the rear wheel and rear frame, or the
front wheel and frame:
\begin{align*}
  \underbrace{\bs{\omega}^C \times \bs{r}^{CO/CN}}_{\text{Rear wheel no-slip}}
  +
  \underbrace{\bs{\omega}^D \times \bs{r}^{DE/CO}}_{\text{DE is fixed in D}}
  & =
  \underbrace{\bs{\omega}^F \times \bs{r}^{FO/FN}}_{\text{Front wheel no-slip}}
  +
  \underbrace{\bs{\omega}^E \times \bs{r}^{DE/FO}}_{\text{DE is fixed in E}}
\end{align*}

This vector expression is linear in the generalized speeds, so resolving it in
a specific frame results in three scalar equations which can be written in
matrix form:
\begin{align*}
  B u & = 0 \\
  B_d u_d + B_i u_i & = 0 \\
  u_d & = -B_d^{-1} B_i u_i
\end{align*}
where $B \in \mathbf{R}^{3 \times 6}$, $u \in \mathbf{R}^{6 \times 1}$, $B_d, B_i \in
\mathbf{R}^{3 \times 3}$, $u_d, u_i \in \mathbf{R}^{3 \times 1}$.

$B_i$ are formed by choosing the columns of $B$ corresponding to the subscript
index of the independent; analogously with $B_d$.  Given three independent
generalized speeds $u_i$, the above constraints provide the means to determine
the dependent speeds $u_d$.  Care should be taken when computing the inverse to
ensure that $B_d$ is well conditioned.

When velocities are differentiated in the inertial frame to obtain
accelerations, the time derivatives of the generalized speeds (both dependent
and independent) appear linearly.  To form the minimal set of ODE's required to obtain
the complete motion of the system, our goal is to determine the time
derivatives of the independent generalized speeds.  To this end, we
differentiate the constraint equations with respect to time:
\begin{align*}
  \frac{d}{dt}\left(B u\right) & = 0 \\
  \dot{B} u + B \dot{u} & = 0 \\
  B_d \dot{u}_d + B_i \dot{u}_i & = -\dot{B} u \\
  \dot{u}_d & = -B_d^{-1} \left(\dot{B} u  + B_i \dot{u}_i \right)
\end{align*}

When forming the angular accelerations of the rigid bodies and the
accelerations of their mass centers, the above relationship must be
incorporated so that the final dynamic equations will satisfy the motion
constraints.

The resulting kinematic differential equations
(involving {\it{all}} generalized speeds) are shown in Table \ref{kindiffs}
Note that expressions for $\dot{q}_7$ and $\dot{q}_8$ in Table \ref{kindiffs}
are defined implicitly for brevity.
\begin{table}[!h]
  \begin{center}
    \begin{tabular}{rclcrcl}
        $\dot{q}_1$ & $=$ & $-(s_3/c_2) u_1 + (c_3/c_2) u_3$ &$\quad$&
        $\dot{q}_5$ & $=$ & $u_5 - u_3$ \\
        $\dot{q}_2$ & $=$ & $c_3 u_1 + s_3 u_3$ &$\quad$&
        $\dot{q}_6$ & $=$ & $s_5 u_1 - c_5 u_2 + u_6$ \\
        $\dot{q}_3$ & $=$ & $s_3 t_2 u_1 + u_2 - c_3 t_2 u_3$ &$\quad$&
        $\dot{q}_7$ & $=$ & $-r_r c_1 (\dot{q}_3 + \dot{q}_4) -
        r_{rt}(s_1 \dot{q}_2 + c_1 c_2 (\dot{q}_3 + \dot{q}_4))$\\
        $\dot{q}_4$ & $=$ & $u_4 - u_2$ & $\quad$ &
        $\dot{q}_8$ & $=$ & $-r_r s_1 (\dot{q}_3 + \dot{q}_4) +
        r_{rt}(c_1 \dot{q}_2 - s_1 c_2 (\dot{q}_3 + \dot{q}_4))$
    \end{tabular}
  \end{center}
  \caption{Kinematic differential equations}
  \label{kindiffs}
\end{table}

\section{Active Forces}

There are two types of active forces that arise in the model: contributing and
non-contributing.  The non-contributing the forces are the ground reaction
forces and the internal reaction forces between each rigid body.  The use of
Kane's method allows for the non-contributing forces to be left out of the
analysis entirely \cite{Kane1985}.  This aspect of Kane's method eliminates a
great deal of algebra that must be done in other methods to eliminate non
contributing forces.  The contributing forces arise only due to gravitational
forces acting at the mass centers, and applied torques between the rear wheel
and rear frame, front wheel and front front fork, and between the main frame
and the fork.


\appendix

The entries of the nonholonomic constraint matrix are:

\begin{align*}
  B_{11} & = c_5 s_5(g_{33} r_f + c_2 c_3 r_{ft}) \\
  B_{12} & = l_s + s_5^2(g_{33} r_f + c_2 c_3 r_{ft}) \\
  B_{13} & = r_{rt} s_2 \\
  B_{14} & = -c_3 (r_r + c_2 r_{rt}) \\
  B_{15} & =   -s_2 r_{ft} - s_5 (g_{31} r_f + l_f) \\
  B_{16} & = c_5(g_{33} r_f + c_2 c_3 r_{ft})\\
  B_{21} & = -l_s + c_3(r_r + c_2 r_{rt}) - c_5^2(g_{33} r_f + c_2 c_3 r_{ft})\\
  B_{22} & =  - c_5 s_5(g_{33} r_f + c_2 c_3 r_{ft}) \\
  B_{23} & = l_r + s_3 (r_r  + c_2 r_{rt}) \\
  B_{24} & = 0 \\
  B_{25} & = c_5 (l_f + r_f g_{31}) + c_2 s_3 r_{ft} \\
  B_{26} & = s_5(g_{33} r_f - c_2 c_3 r_{ft}) \\
  B_{31} & = s_5 l_f + s_2 (r_{ft} - r_{rt}) + (c_2 s_3 c_5 s_5 - s_2 s_5^2) r_{ft}\\
  B_{32} & =  -l_r - c_5 l_f + s_5 (s_2 c_5 + c_2 s_3 s_5) r_{ft} \\
  B_{33} & =  0 \\
  B_{34} & =  -s_3 (r_r + c_2 r_{rt}) \\
  B_{35} & =  0 \\
  B_{36} & =  -g_{31} r_f + (c_2 s_3 c_5 - s_2 s_5) r_{ft}
\end{align*}


\begin{comment}
Here is the B matrix, but it is too wide to fit in portrait mode.
\begin{align*}
  \left[
  \begin{array}{cccccc}
    c_5 s_5(g_{33} r_f + c_2 c_3 r_{ft}) &
    l_s + s_5^2(g_{33} r_f + c_2 c_3 r_{ft}) &
    r_{rt} s_2 &
    -c_3 (r_r + c_2 r_{rt}) &
    -s_2 r_{ft} - s_5 (g_{31} r_f + l_f) &
    c_5(g_{33} r_f + c_2 c_3 r_{ft})
    \\
    -l_s + c_3(r_r + c_2 r_{rt}) - c_5^2(g_{33} r_f + c_2 c_3 r_{ft}) &
    - c_5 s_5(g_{33} r_f + c_2 c_3 r_{ft}) &
    l_r + s_3 (r_r  + c_2 r_{rt}) &
    0 &
    c_5 (l_f + r_f g_{31}) + c_2 s_3 r_{ft} &
    s_5(g_{33} r_f - c_2 c_3 r_{ft})
    \\
    s_5 l_f + s_2 (r_{ft} - r_{rt}) + (c_2 s_3 c_5 s_5 - s_2 s_5^2) r_{ft} &
    -l_r - c_5 l_f + s_5 (s_2 c_5 + c_2 s_3 s_5) r_{ft} &
    0 &
    -s_3 (r_r + c_2 r_{rt}) &
    0 &
    -g_{31} r_f + (c_2 s_3 c_5 - s_2 s_5) r_{ft}
  \end{array}
  \right]
\end{align*}
\end{comment}


\bibliographystyle{plain}
\bibliography{../../../../../Documents/references}

\end{document}
