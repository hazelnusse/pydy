\documentclass[letterpaper,11pt]{article}
\usepackage[margin=1in,centering]{geometry}
\usepackage{fancyhdr}
\usepackage{amsmath}
\usepackage{amssymb}
\newcommand{\bs}[1]{ \boldsymbol{ #1 } }
\newcommand{\ud}{\mathrm{d}}
\pagestyle{fancy}
\fancyhead[L]{Benchmark bicycle model equations of motion, Dale L. Peterson}
\fancyhead[R]{\today}
\fancyfoot[L,R]{}
\fancyfoot[C]{\thepage}
\begin{document}
\begin{abstract}

  Nonlinear equations of motion for a benchmark bicycle are presented in
  symbolic form.  They are validated numerically against Meijaard et al. and
  Basu-Mandal et al.  Numerical results match to at least 14 decimal places.

\end{abstract}

\section*{Model description}
Consider the following four rigid bodies: a rear wheel $C$, a frame with
rigidly attached rider $D$, a front fork/handlebar assembly $E$, and a front
wheel $F$.
The following eight (non-minimal) generalized coordinates are defined:
\begin{table}[h]
  \begin{center}
    \begin{tabular}{rl}
      $q_1:$ & Rear frame yaw \\
      $q_2:$ & Rear frame lean \\
      $q_3:$ & Rear frame pitch \\
      $q_4:$ & Rear wheel angle relative to rear frame \\
      $q_5:$ & Steer angle relative to rear frame \\
      $q_6:$ & Front wheel angle relative to front fork \\
      $q_7:$ & Rear wheel contact point position in $\bs{n}_1$ direction \\
      $q_8:$ & Rear wheel contact point position in $\bs{n}_2$ direction
    \end{tabular}
  \end{center}
\end{table}

The bicycle frame with rigidly attached rider $D$ are oriented with Euler 3-1-2
($q_1, q_2, q_3$) angles (identical to the ($\psi, \phi, \theta_B$) Euler angles
in \cite{Meijaard2007}).  The rear wheel $R$ is oriented relative to $D$ by
an angle $q_4$ ($\theta_R$ in \cite{Meijaard2007}), the front fork relative to
$D$ by an angle $q_5$ ($\phi$ in \cite{Meijaard2007}), and the front wheel
relative to the front fork $E$ by an angle $q_6$ ($\theta_F$ in
\cite{Meijaard2007}).  Explicit direction cosine matrices are provided in the
appendix.

Define six generalized speeds:
\begin{align*}
  u_i & \triangleq \bs{\omega}^D \cdot \bs{d}_i \qquad (i = 1, 2, 3)\\
  u_4 & \triangleq \bs{\omega}^C \cdot \bs{d}_2\\
  u_5 & \triangleq \bs{\omega}^E \cdot \bs{d}_3\\
  u_6 & \triangleq \bs{\omega}^F \cdot \bs{e}_2
\end{align*}

These generalized speed definitions follow the guidelines of \cite{Mitiguy1996}
for generating `efficient' equations of motion.  As it turns out, they also
simplify the formulation of the motion constraints imposed by pure rolling.  As
will be made clear in the next section, only three of these six generalized
speeds can be treated as independent.  The resulting angular velocities (with
respect to the inertial frame) of each of the four rigid bodies are:
\begin{align*}
  \bs{\omega}^D & = u_1 \bs{d}_1 + u_2 \bs{d}_2 + u_3 \bs{d}_3 \\
  \bs{\omega}^C & = u_1 \bs{d}_1 + u_4 \bs{d}_2 + u_3 \bs{d}_3 \\
  \bs{\omega}^E & = u_1 \bs{d}_1 + u_2 \bs{d}_2 + u_5 \bs{d}_3 \\
                & = (c_5 u_1 + s_5 u_2)\bs{e}_1 + (-s_5 u_1 + c_5 u_2) \bs{e}_2
                    + u_5 \bs{e}_3 \\
  \bs{\omega}^F & = (c_5 u_1 + s_5 u_2)\bs{e}_1 + u_6 \bs{e}_2
                    + u_5 \bs{e}_3
\end{align*}

\section*{Motion constraints}

Before formulating the motion constraints, some geometric considerations
must be addressed.  The unit vector lying in the plane of the front wheel that
can be thought of as pointing from the front wheel center torwards the front
contact point is formed by taking the projection of the $\bs{n}_3$ unit vector
onto the plane of the front wheel, and normalizing to make it unit length:
\begin{align*}
  \bs{g}_3 & = (\bs{n}_3 - (\bs{n}_3 \cdot \bs{e}_2) \bs{e}_2) / |(\bs{n}_3 -
  (\bs{n}_3 \cdot \bs{e}_2) \bs{e}_2)| \\
  & = ((s_2 s_5 - c_2 s_3 c_5) \bs{e}_1 + c_2 c_3 \bs{e}_3) / \sqrt{(s_2 s_5 -
  c_2 s_3 c_5)^2 + (c_2 c_3)^2}
\end{align*}

Given rear frame lean $q_2$, rear frame pitch $q_3$, and the steer angle
$q_5$, the front frame lean $q_9$ and front frame pitch $q_{10}$ are determined by:
\begin{align*}
    q_9 & = \arcsin(\bs{e}_2 \cdot \bs{n}_3) \\
    & = \arcsin(s_2 c_5 + c_2 s_3 s_5) \\
    q_{10} & = \arcsin(-\bs{e}_1 \cdot \bs{g}_3) \\
           & = \arcsin((c_2 s_3 c_5 - s_2 s_5) / \sqrt{(s_2 s_5 -
                c_2 s_3 c_5)^2 + (c_2 c_3)^2})
\end{align*}

For purpose of determining ground reaction forces, the vectors in the ground
plane that are aligned with, and perpendicular to, the front and rear wheel
tracks, are:

\begin{tabular}{rcc}
  & Rear wheel & Front wheel \\
  Tangent to path: & $\bs{a}_1$ & blah \\
  Normal to path: & $\bs{a}_2$ & blah
\end{tabular}

%\begin{align*}
%  \bs{g}_1 & = \bs{e}_2 \times \bs{g}_3 \\
%  & = ( c_2 c_4 \bs{e}_1 + (s_2 s_5 - c_2 c_5 s_4) \bs{e}_3) / \sqrt{(s_2 s_5 -
%  c_2 c_5 s_4)^2 + (c_2 c_4)^2}
%\end{align*}
%The unit vector parallel to the ground plane and perpendicular to the
%front wheel track, expressed in the yawed frame $A$, is:
%\begin{align*}
%  \bs{h}_2 & = \bs{n}_3 \times \bs{g}_1 \\
%(c2*c5 - s2*s4*s5)*a1> + c4*s5*a2>
%  & = ( (c_2 c_5 - s_2 s_4 s_5) \bs{a}_1 + c_4 s_5 \bs{a}_2) / \sqrt{(s_2 s_5 -
%  c_2 c_5 s_4)^2 + (c_2 c_4)^2}
%\end{align*}


The nonholonomic constraints may be written by formulating the velocity of the
top of the steer axis $DE$ in two separate ways, and requiring that both be
equal.  The first approach makes use of the rear wheel and rear frame, while
the second approach makes use of the front wheel and frame.
\begin{align*}
  \bs{v}^{DE} & = \bs{\omega}^C \times \bs{r}^{C^* / CN} + \bs{\omega}^D
  \times \bs{r}^{DE / C^*} \\
  & = (u_1 \bs{d}_1 + u_4 \bs{d}_2 + u_3 \bs{d}_3) \times (-rr \bs{b}_3) + (u_1
  \bs{d}_1 + u_2 \bs{d}_2 + u_3 \bs{d}_3) \times (l_r \bs{d}_1) \\
  & = -r_r u_4 \bs{b}_1 + (r_r c_4 u_1 + r_r s_4 u_3) \bs{b}_2
\end{align*}

\bibliographystyle{plain}
\bibliography{../../../../../Documents/references}

\end{document}
