\documentclass[letterpaper,11pt]{article}
\usepackage[margin=1in,centering]{geometry}
\usepackage{fancyhdr}
\usepackage{amsmath}
\usepackage{amssymb}
\newcommand{\bs}[1]{ \boldsymbol{ #1 } }
\newcommand{\ud}{\mathrm{d}}
\pagestyle{fancy}
\fancyhead[L]{Gyrostat equations of motion, Dale L. Peterson}
\fancyhead[R]{\today}  % page number on the right
\fancyfoot[L,R]{}  %  No footer on left, center or right, on even or odd pages
\fancyfoot[C]{\thepage}
\begin{document}
\begin{abstract}

  Equations of motion for a gyrostat comprised of two rigid bodies (a carrier
  with a lateral plane of symmetry and an axially symmetric a rotor) are
  derived in two different ways.  In the first approach, each rigid body is
  treated separately as each having their own mass and inertia.  In the second
  approach, the symmetry of the rotor is used to combine the inertias in such a
  way that simplifies the derivation of the equations of motion.  It is shown
  that the two approaches are equivalent.  For generality, the axis of rotation
  of the rotor is not aligned with a principal axis of the carrier.

\end{abstract}

\section*{Model Description}
Consider a rigid body $A$, of mass $m_a$, along with a dextral set of unit
vectors $\bs{a}_1$, $\bs{a}_2$, $\bs{a}_3$, fixed to $A$, with $\bs{a}_3
\triangleq \bs{a}_1 \times \bs{a}_2$.  Let $A^*$ denote the center of mass of
$A$.  Let the inertia dyadic of $A$ relative to its mass center $A^*$ for be:
\begin{align*}
  \bs{I}^{A/A^*} & = I_{11}\bs{a}_1\bs{a}_1 + I_{22}\bs{a}_2\bs{a}_2 +
  I_{33}\bs{a}_3\bs{a}_3 + I_{13}\bs{a}_1\bs{a}_3 + I_{13}\bs{a}_3\bs{a}_1
\end{align*}
The plane spanned by the $\bs{a}_1$ and $\bs{a}_3$ unit vectors is a plane of
symmetry.

Consider a second rigid body $B$, of mass $m_b$, along with a dextral set of unit
vectors $\bs{b}_1$, $\bs{b}_2$, $\bs{b}_3$, fixed to $B$, with $\bs{b}_3
\triangleq \bs{b}_1 \times \bs{b}_2$.  Let $B^*$ denote the center of mass of
$B$.  Let the inertia dyadic of $B$ relative to its mass center  $B^*$ for be:
\begin{align*}
  \bs{I}^{B/B^*} & = I\bs{b}_1\bs{b}_1 + J\bs{b}_2\bs{b}_2 + I\bs{b}_3\bs{b}_3
\end{align*}
$B$ is axially symmetric about the $\bs{b}_2$ axis.

Let $B$ be rigidly fixed to $A$ such that the mass center $B^*$ is located
relative to $A^*$ by:
\begin{align*}
  \bs{r}^{B^* / A^*} & = l_1 \bs{a}_1 + l_3 \bs{a}_3
\end{align*}
Where $l_1, l_3$ are real constants.  Clearly, $B^*$ lies in the plane of
symmetry of the rigid body $A$.

Denote the center of mass of the system as $AB^*$. The position of $A^*$ and
$B^*$ relative to the system mass center $AB^*$ is:

\begin{align*}
  \bs{r}^{A^* / AB^*} & = -\frac{m_b}{m_a+m_b}(l_1 \bs{a}_1 + l_3 \bs{a}_3) \\
  \bs{r}^{B^* / AB^*} & =  \frac{m_a}{m_a+m_b}(l_1 \bs{a}_1 + l_3 \bs{a}_3)
\end{align*}

$B$ is oriented relative to $A$ by means of a revolute joint whose axis is
aligned with the $\bs{a}_2$ and passes through $B^*$.  Because of the inertial
symmetry of the rotor, the central inertia dyadic of $B$ can be expressed
equivalently in the $A$ frame:

\begin{align*}
  \bs{I}^{A/A^*} & = I\bs{a}_1\bs{a}_1 + J\bs{a}_2\bs{a}_2 + I\bs{a}_3\bs{a}_3
\end{align*}

A couple of torque $T$ is applied between $A$ and $B$ about the axis of
rotation $\bs{a}_2$.  If $A$ were held fixed, a positive torque $T$ would cause
$B$ to rotate relative to $A$ in a counterclockwise direction about the
$\bs{a}_2$ axis.

\section*{Equations of motion}
\subsection*{Kinematics}

The system is completely described by seven generalized coordinates: three
which locate the system in inertial space, three which orient $A$ in inertial
space, and one which orients $B$ relative to $A$.

The orientation of $A$ relative to the inertial frame $N$ is given by a
sequence of rotations ($q_1$, $q_2$, $q_3$) 3-1-2 Euler angles.  The direction
cosine matrix relating the unit vectors fixed in $N$ and $A$ is shown in Table
\ref{directioncosines}.

\begin{table}[!h]
  \begin{center}
    \begin{tabular}{l|ccc}
      & $\bs{a}_1$ & $\bs{a}_2$ & $\bs{a}_3$ \\
      \hline
      $\bs{n}_1$ & $c_1c_3 - s_1s_2s_3$ & $-s_1c_2$ & $c_1s_3 + c_3s_1s_2$ \\
      $\bs{n}_2$ & $s_1c_3 + c_1s_2s_3$ & $ c_1c_2$& $s_1s_3 - c_1c_3s_2$ \\
      $\bs{n}_3$ & $-c_2s_3$ & $s_2$& $c_2c_3$
    \end{tabular}
  \end{center}
  \caption{Direction cosine matrix relating unit vectors of $A$ and $N$}
  \label{directioncosines}
\end{table}

$B$ is oriented relative to $A$ by first aligning $\bs{b}_i$ with $\bs{a}_i
\:\: (i = 1,2,3)$ then performing a right handed rotation of $B$ by an angle
$q_4$ about the $\bs{a}_2$ axis.

Let the mass center $AB^*$ be located relative to the inertial origin $N^*$ by:
\begin{align*}
  \bs{r}^{AB^*/N^*} & = q_5 \bs{n}_1 + q_6 \bs{n}_2 + q_7 \bs{n}_3
\end{align*}

For sake of symbolic brevity, introduce the following terms:
\begin{align*}
  l_{1a} & = -\frac{l_1 m_b}{m_a + m_b} &
  l_{3a} & = -\frac{l_3 m_b}{m_a + m_b} \\
  l_{1b} & = \frac{l_1 m_a}{m_a + m_b} &
  l_{3b} & = \frac{l_3 m_a}{m_a + m_b} \\
\end{align*}

Let $\omega^A$ and $\omega^B$ denote the angular velocity of $A$ and $B$
relative to the inertial frame, respectively.  Let $v^{A^*}$, $v^{B^*}$, and $v^{AB^*}$ denote the
velocity of $A^*$, $B^*$, and $AB^*$ relative to the inertial frame,
respectively.

Define the following generalized speeds:
\begin{align}
  u_i & \triangleq \bs{\omega}^A \cdot \bs{a}_i  \qquad  (i = 1,2,3) \label{u_defs1} \\
  u_4 & \triangleq \bs{\omega}^B \cdot \bs{a}_2 \label{u_defs2} \\
  u_i & \triangleq \bs{v}^{AB^*} \cdot \bs{n}_{i-4} \qquad  (i = 5,6,7) \label{u_defs3}
\end{align}
The above generalized speeds result in the following angular velocities and
velocities:
\begin{align*}
  \bs{\omega}^A & = u_1 \bs{a}_1 + u_2 \bs{a}_2 + u_3 \bs{a}_3 \\
  \bs{\omega}^B & = u_1 \bs{a}_1 + u_4 \bs{a}_2 + u_3 \bs{a}_3 \\
  \bs{v}^{AB^*} & = u_5 \bs{n}_1 + u_6 \bs{n}_2 + u_7 \bs{n}_3 \\
  \bs{v}^{A^*}  & = \bs{v}^{AB^*} + \bs{\omega}^A \times \bs{r}^{A^*/AB^*} \\
  & = u_5 \bs{n}_1 + u_6 \bs{n}_2 + u_7 \bs{n}_3 +l_{3a}u_2\bs{a}_1 +
  (l_{1a}u_3 - l_{3a}u_1)\bs{a}_2 - l_{1a}u_2\bs{a}_3 \\
  \bs{v}^{B^*}  & = \bs{v}^{AB^*} + \bs{\omega}^A \times \bs{r}^{B^*/AB^*} \\
  & = u_5 \bs{n}_1 + u_6 \bs{n}_2 + u_7 \bs{n}_3 +l_{3b}u_2\bs{a}_1 +
  (l_{1b}u_3 - l_{3b}u_1)\bs{a}_2 - l_{1b}u_2\bs{a}_3
\end{align*}
All of the above velocities are relative to the inertial frame, and of particular
interest here is that the angular velocity expressions for $A$ and $B$ only
differ by the $\bs{a}_2$ component, which is the axis of rotation of $B$
relative to $A$.

By inspection of the angular velocity expressions for $A$ and $B$, and the
velocity expressions for $A^*$ and $B^*$, the partial angular velocities and
partial velocities can be determined.  They are simply the vector coefficients
of the generalized speeds, which appear linearly in the velocity and angular
velocity expressions.  They are shown in Table \ref{partialvelocities}.

\begin{table}[!h]
  \begin{center}
    \begin{tabular}{lllllll}
      $\bs{\omega}^A_1 = \bs{a}_1$ & $\bs{\omega}^A_2 = \bs{a}_2$ &
      $\bs{\omega}^A_3 = \bs{a}_3$ & $\bs{\omega}^A_4 = \bs{0}$ &
      $\bs{\omega}^A_5 = \bs{0}$ & $\bs{\omega}^A_6 = \bs{0}$ &
      $\bs{\omega}^A_7 = \bs{0}$ \\
%
      $\bs{\omega}^B_1 = \bs{a}_1$ & $\bs{\omega}^B_2 = \bs{0}$ &
      $\bs{\omega}^B_3 = \bs{a}_3$ & $\bs{\omega}^B_4 = \bs{a}_2$ &
      $\bs{\omega}^B_5 = \bs{0}$ & $\bs{\omega}^B_6 = \bs{0}$ &
      $\bs{\omega}^B_7 = \bs{0}$ \\
%
      $\bs{v}^{AB^*}_1 = \bs{0}$ & $\bs{v}^{AB^*}_2 =
      \bs{0}$ & $\bs{v}^{AB^*}_3 = \bs{0}$ &
      $\bs{v}^{AB^*}_4 = \bs{0}$ & $\bs{v}^{AB^*}_5 = \bs{n}_1$ & $\bs{v}^{AB^*}_6
      = \bs{n}_2$ & $\bs{v}^{AB^*}_7 = \bs{n}_3$ \\
%
      $\bs{v}^{A^*}_1 = -l_{3a}\bs{a}_2$ & $\bs{v}^{A^*}_2 =
      l_{3a}\bs{a}_1 - l_{1a}\bs{a}_3$ & $\bs{v}^{A^*}_3 = l_{1a} \bs{a}_2$ &
      $\bs{v}^{A^*}_4 = \bs{0}$ & $\bs{v}^{A^*}_5 = \bs{n}_1$ & $\bs{v}^{A^*}_6
      = \bs{n}_2$ & $\bs{v}^{A^*}_7 = \bs{n}_3$ \\
%
        $\bs{v}^{B^*}_1 = -l_{3b}\bs{a}_2$ & $\bs{v}^{B^*}_2 = l_{3b}\bs{a}_1 -
        l_{1b}\bs{a}_3$ & $\bs{v}^{B^*}_3 = l_{1b} \bs{a}_2$ & $\bs{v}^{B^*}_4
        = \bs{0}$ & $\bs{v}^{B^*}_5 = \bs{n}_1$ & $\bs{v}^{B^*}_6 = \bs{n}_2$ &
        $\bs{v}^{B^*}_7 = \bs{n}_3$
%
    \end{tabular}
  \end{center}
  \caption{Partial angular velocities and partial velocities}
  \label{partialvelocities}
\end{table}

These definitions, along with (\ref{u_defs1}, \ref{u_defs2}, \ref{u_defs3})
yield the following kinematic differential equations:
\begin{align*}
  \dot{q}_1 &=  -s_3u_1/c_2 + c_3u_3/c_2\\
  \dot{q}_2 &=  c_3u_1 + s_3u_3 \\
  \dot{q}_3 &=  t_2s_3u_1 + u_2 - t_2c_3u_3 \\
  \dot{q}_4 &= -u_2 +  u_4 \\
  \dot{q}_5 &= u_5 \\
  \dot{q}_6 &= u_6 \\
  \dot{q}_7 &= u_7
\end{align*}
While these equations are necessary for kinematics, they aren't too interesting
because all the coordinates are ignorable with respect to the dynamic
differential equations (no coordinates appear in the dynamic equations of
motion).

\subsection*{Dynamics}
There are two types of active forces acting on the system, the first arising
from gravity, the second arising from the applied torque between the two
bodies.
\begin{align*}
  \bs{G}_A & = m_ag\bs{n}_3 \\
  \bs{G}_B & = m_bg\bs{n}_3 \\
  \bs{T}_A & = - T \bs{a}_2 \\
  \bs{T}_B & = T \bs{a}_2
\end{align*}

In order to compute the inertia forces of the system, we need to form the
angular accelerations of $A$ and $B$ as well as accelerations of the points
$AB^*$, $A^*$, and $B^*$. The angular accelerations relative to the inertial
frame are:
\begin{align*}
  \bs{\alpha}^A & = \dot{u}_1 \bs{a}_1 + \dot{u}_2 \bs{a}_2 + \dot{u}_3 \bs{a}_3\\
  \bs{\alpha}^B & =(u_2u_3 - u_3u_4 + \dot{u}_1)\bs{a}_1 + \dot{u}_4\bs{a}_2 + (u_1u_4 -
  u_1u_2 + \dot{u}_3)\bs{a}_3
\end{align*}
The accelerations of $AB^*$, $A^*$, and $B^*$, relative to the inertial
origin are:
\begin{align*}
  \bs{a}^{AB^*} &= \dot{u}_5\bs{n}_1 + \dot{u}_6\bs{n}_2 + \dot{u}_7\bs{n}_3
  \\
  \bs{a}^A &= \dot{u}_5\bs{n}_1 + \dot{u}_6\bs{n}_2 + \dot{u}_7\bs{n}_3
  + (l_{3a}(\dot{u}_2 + u_1u_3) - l_{1a}(u_2^2 + u_3^2))\bs{a}_1 \\
  &+ (l_{1a}(\dot{u}_3 + u_1u_2) + l_{3a}(u_2u_3 - \dot{u}_1))\bs{a}_2
  - (l_{1a}(\dot{u}_2 + u_1u_3) + l_{3a}(u_1^2 + u_2^2))\bs{a}_3
  \\
  \bs{a}^B &= \dot{u}_5\bs{n}_1 + \dot{u}_6\bs{n}_2 + \dot{u}_7\bs{n}_3
  + (l_{3b}(\dot{u}_2 + u_1u_3) - l_{1b}(u_2^2 + u_3^2))\bs{a}_1 \\
  &+ (l_{1b}(\dot{u}_3 + u_1u_2) + l_{3b}(u_2u_3 - \dot{u}_1))\bs{a}_2
  - (l_{1b}(\dot{u}_2 + u_1u_3) + l_{3b}(u_1^2 + u_2^2))\bs{a}_3
\end{align*}

The inertia force and inertia torque for the rigid bodies $A$ and $B$ are:
\begin{align*}
  \bs{R}^*_A  =& -m_a \bs{a}^{A^*} \\
  \bs{T}^*_A  =& -\bs{\alpha}^A \cdot \bs{I}^{A/A^*} - \bs{\omega}^A \times
  \bs{I}^{A/A^*} \cdot \bs{\omega}^A \\
  = &-(I_{11}\dot{u}_1 + I_{13}\dot{u}_3)\bs{a}_1 - I_{22}\dot{u}_2\bs{a}_2 -
  (I_{13}\dot{u}_1 + I_{33}\dot{u}_3)\bs{a}_3 \\
   + & (I_{13}u_1u_2 + (I_{33} - I_{22})u_2u_3)\bs{a}_1 +
  (I_{13}u_1^2 + (I_{33}- I_{11})u_1u_3 - I_{13}u_3^2)\bs{a}_2 \\
  + & ((I_{11} - I_{22})u_1u_2 + I_{13}u_2u_3)\bs{a}_3 \\
  \bs{R}^*_B = & -m_b \bs{a}^{B^*} \\
  \bs{T}^*_B = & -\bs{\alpha}^B \cdot \bs{I}^{B/B^*} - \bs{\omega}^B \times
  \bs{I}^{B/B^*} \cdot \bs{\omega}^B \\
  = & -I(u_2u_3 - u_3u_4 + \dot{u}_1)\bs{a}_1 - J\dot{u}_4\bs{a}_2 - I(u_1u_4 - u_1u_2 +
  \dot{u}_3)\bs{a}_3 \\
  + & (Ju_3u_4 - Iu_3u_4)\bs{a}_1 + (Iu_1u_4 - Ju_1u_4)\bs{a}_3 \\
  = &-(I\dot{u}_1 - Ju_3u_4 + Iu_2u_3)\bs{a}_1 - J\dot{u}_4\bs{a}_2 -
  (I\dot{u}_3 + Ju_1u_4 - Iu_1u_2)\bs{a}_3
\end{align*}

The first four (orientation related) dynamic differential equations are:
\begin{align*}
  -\left(I + I_{11} + \frac{m_al_3^2m_b^2 + m_bl_3^2m_a^2}{(m_a +
  m_b)^2}\right)\dot{u}_{1}
  + \left(-I_{13} + \frac{l_1l_3m_am_b^2 + l_1l_3m_bm_a^2}{(m_a + m_b)^2}\right)\dot{u}_{3}
  & = \\
  \left(I_{13} - \frac{l_1l_3m_am_b^2 + l_1l_3m_bm_a^2}{(m_a + m_b)^2}\right)u_1u_2 +
  \left(I + I_{33} - I_{22} - \frac{m_al_3^2m_b^2 + m_bl_3^2m_a^2}{(m_a +
  m_b)^2}\right)u_2u_3 &- Ju_3u_4 &
  \\
  \left(-I_{22} - \frac{m_al_1^2m_b^2 + m_al_3^2m_b^2 + m_bl_1^2m_a^2 +
  m_bl_3^2m_a^2}{(m_a + m_b)^2}\right) \dot{u}_2
  & = \\
  T + \left(-I_{13} + \frac{l_1l_3m_am_b^2 + l_1l_3m_bm_a^2}{(m_a +
  m_b)^2}\right)u_1^2
  + \left(I_{13} - \frac{l_1l_3m_am_b^2 + l_1l_3m_bm_a^2}{(m_a +
  m_b)^2}\right)u_3^2 &
  \\
  +\left(I_{11} - I_{33} + \frac{m_al_3^2m_b^2 + m_bl_3^2m_a^2 -
  m_al_1^2m_b^2 - m_bl_1^2m_a^2}{(m_a + m_b)^2}\right)u_1u_3 &
  \\
  \left(-I_{13} + \frac{l_1l_3m_am_b^2 + l_1l_3m_bm_a^2}{(m_a + m_b)^2}\right)
  \dot{u}_1 -
  \left(I + I_{33} + \frac{m_al_1^2m_b^2 + m_bl_1^2m_a^2}{(m_a + m_b)^2}
  \right) \dot{u}_3
  & = \\
  \left(I_{22} - I - I_{11} + \frac{m_al_1^2m_b^2 +
  m_bl_1^2m_a^2}{(m_a + m_b)^2}\right)u_1u_2 + \left(-I_{13} + \frac{l_1l_3m_am_b^2 + l_1l_3m_bm_a^2}{(m_a +
  m_b)^2}\right)u_2u_3 &+ Ju_1u_4\\
  -J\dot{u}_4 &=  -T
\end{align*}
The three equations of motion associated with translation are trivial:
$\dot{u}_5 = 0$, $\dot{u}_6 = 0$, $\dot{u}_7 = -g$.
\end{document}
